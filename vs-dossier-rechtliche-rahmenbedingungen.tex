% !TeX encoding = UTF-8
% !TeX spellcheck = de_DE
% !TeX program = pdflatex

\documentclass[
10pt,
a4paper,
twoside,								% oneside oder twoside?
titlepage=false,							% Extra Titelseite?
draft=false								% Entwurfsmodus?
]{scrartcl}


\usepackage[top=2.5cm, bottom=2.5cm, left=2.5cm, right=2.5cm]{geometry}  % hiermit ohne Rand für Bindung
%\usepackage[top=2.5cm, bottom=2.5cm, left=3cm, right=2cm]{geometry}  % hiermit mit Rand für Bindung
%\usepackage{fullpage}
\usepackage{layout}							% für \layout{} → Darstellung der aktuellen Ränder und Co.

\usepackage[utf8]{inputenc}
\usepackage[T1]{fontenc}					% für echte Umlaute und mehr…
\usepackage{lmodern}						% schönere Schrift, v.a in PDFs
\renewcommand{\familydefault}{\sfdefault}
\usepackage[ngerman]{babel}					% deutsche Sprachvariante (Inhaltsverzeichnis, …)

\usepackage{varioref}
\usepackage{textcomp}						% für €-Symbol http://www.theiling.de/eurosym.html
\usepackage{fancyvrb}						% schönrer Verbatim-Umgebgunen, für die Nutzung von latexdiff
\usepackage{enumerate}						% anpassen von Aufzählungslisten
\usepackage{graphicx}						% für Grafiken
\usepackage{tipa}							% für IPA Zeichen
\usepackage[usenames,dvipsnames]{xcolor}	% benannte Farben, https://en.wikibooks.org/wiki/LaTeX/Colors

% Kopf- und Fußzeilen:
%%   http://www2.informatik.hu-berlin.de/~piefel/LaTeX-PS/Archive-2004/V07-footnote.pdf
\usepackage{fancyhdr}
\usepackage{lastpage}

%TODO scrjura-Funktionen noch verwenden
\usepackage[
juratotoc,								% Inhaltsverzeichnis über die §§
ref=long								% Verweise ausgeschrieben, z.B. »§ 314 Absatz 2 Satz 2« (Standardwert)
]{scrjura}
%   http://www.komascript.de/node/1404


% muss als letztes Paket geladen werden:
\usepackage{hyperref}
% und das hier noch eins später laden:
%\usepackage[hyphenbreaks]{breakurl}  % -> nicht für pdflatex

\KOMAoptions{parskip=half}					% Alle Absätze vorne beginnen lassen.


% Seite:
\flushbottom								% auch letzte Zeile einer Seite soll immer auf gleicher Höhe sein (→ twoside)
\pagestyle{fancy}

%\renewcommand{\chaptermark}[1]{\markboth{\MakeUppercase{\chaptername\ \thechapter.\ #1}}{}}
%\renewcommand{\chaptermark}[1]{ \markboth{#1}{} }
%\renewcommand{sectionmark}[1]{\markboth{\MakeUppercase{\sectionname\ \thesection.\ #1}}{}}
\renewcommand{\sectionmark}[1]{\markboth{\MakeUppercase{#1}}{}}

% Kopf- und Fußzeilen:
\lhead
[ \thepage ]
{ \leftmark }
\chead[  ]{  }
\rhead
[ \leftmark ]
{ \thepage }

\renewcommand{\headrulewidth}{0.5pt}
\renewcommand{\footrulewidth}{0.0pt}

\lfoot
%	[\thepage\ | \pageref{LastPage}]
[]
{  }
\cfoot[  ]{  }
\rfoot
[  ]
%{\pageref{LastPage} | \thepage}
{}


%TODO Schöne Lösung für die Darstellung (Bildschirm + Druck) von Hyperlinks finden.
%   https://en.wikibooks.org/wiki/LaTeX/Hyperlinks#Customization
\definecolor{hellgrau}{RGB}{240,240,240}
\definecolor{linkblau}{RGB}{0,60,60}
\definecolor{hellblau}{RGB}{0,90,90}
\hypersetup{
	%hidelinks,							% Links im PDF unsichtbar zu machen.
	%pdfborderstyle={/S/U/W 0.7},
	breaklinks=true,
	colorlinks=true,
	linkcolor=linkblau,
	linkbordercolor=linkblau,
	citecolor=linkblau,
	filecolor=linkblau,
	urlcolor=linkblau,
	linktoc=all,
	pdftitle={Rechtliche Grundlagen und Rahmenbedingungen für die neue Studierendenvertretung},    % title
	pdfauthor={Simon Lüke, Barbara Körner},     % author
	pdfsubject={Für die StuVe der Uni Ulm sowie allgemein für die Verfasste Studierendenschaft in BadenWürttemberg},   % subject of the document
	pdfcreator={Simon Lüke},   % creator of the document
	pdfproducer={Simon Lüke}, % producer of the document
	pdfkeywords={VS} {StuVe} {Verfasst} {Uni} {Ulm} {Studierendenschaft} {Studium} {Studierendenvertretung} {LHG} {Landeshochschulgesetz} {BaWü} {Baden-Württemberg}, % list of keywords
	%bookmarks=false,
	unicode=true,
	pdftoolbar=true,        % show Acrobat’s toolbar?
	pdfmenubar=true,        % show Acrobat’s menu?
	pdffitwindow=false,     % window fit to page when opened
	pdfstartview={FitV},    % fits the width of the page to the window
	pdfnewwindow=true      % links in new window
}


% Inhaltsverzeichnis
%\setcounter{secnumdepth}{5}
%\setcounter{tocdepth}{5}




\begin{document}

\titlehead{\href{http://stuve.uni-ulm.de}{StuVe – StudierendenVertretung, uulm}}


\subject{}

\title{Rechtliche Grundlagen und Rahmenbedingungen für die neue Studierendenvertretung}

\subtitle{Für die StuVe der Uni Ulm\thanks{\href{mailto:stuve.kontakt@uni-ulm.de}{stuve.kontakt@uni-ulm.de}}~ sowie allgemein\\für die Verfasste Studierendenschaft in Baden-Württemberg}

\author{Simon Lüke, Barbara Körner}


\date{\texttt{version03}\\\bigskip{\normalsize inhaltlicher Stand vom 11. Januar 2015\\zuletzt kompiliert am \today}}

\maketitle
\thispagestyle{empty}

\tableofcontents

\vfill

\begin{center}
	\textit{Lizenz: zur freien Verfügung. Bitte weitergeben und weiterentwickeln, evtl.~ist dabei die Nennung der bisher beteiligten und der Verweis auf die Quelle sinnvoll.}

	\textit{Den jeweils aktuellen Stand gibt es hier:\\}
	\url{https://github.com/semaphor/vs-dossier-rechtlihe-rahmenbedingungen}
\end{center}



\newpage
\thispagestyle{empty}

\section*{Abkürzungen (an der Uni Ulm)}

%TODO Liste alphabetisch sortieren oder gleich ein entsprechendes Paket für Abkürzungen benutzen.

\begin{itemize}
	\item AStA: Allgemeiner Studierendenausschuss, siehe \nameref{sec:Glossar}.
	\item BO: BeitragsOrdnung
	\item FO: FinanzOrdnung
	\item FS: FachbereichSvertretung, umgangssprachlich auch „Fachschaft“.
	\item FSR: FachSchaftenRat
	\item GO: GeschäftsOrdnung, kann aufgeschrieben oder auch einfach „tradiert“ sein (Gewohnheitsrecht).
	\item LHG: LandesHochschulGesetz
	\item OS: OrganisationsSatzung
	\item StEx: StudierendenExekutive
	\item StuPa: StudierendenParlament
	\item StuVe: StudierendenVertretung, Ulmer Bezeichnung für die Gesamtheit aller Organe, Ebenen, Aktiven der Verfassten Studierendenschaft; schönerweise war das Ulmer \href{https://de.wikipedia.org/wiki/Unabh\%C3\%A4ngige_Studierendenschaft}{U-Modell} auch schon so benannt.
	\item Uni: Universität, Hochschulform
	\item VerfStudG: Verfasste-Studierendenschafts-Gesetz
	\item VS: Verfasste Studierendenschaft
	\item WO: WahlOrdnung
\end{itemize}

\section*{ToDo für dieses Dokument}

\begin{itemize}
	\item Weitere Stichworte im Glossar ergänzen.
	\item Zusammenfassungen für die Stichworte erstellen, weiter bearbeiten. Sonst einfach \textit{\%TODO} stehen lassen. Wichtig sind sicher noch Zusammenfassungen zu:
	\begin{itemize}
		\item Öffentliches Recht
		\item Ein paar Worte zum doch sehr recht allgemeinen Anspruch bei den Aufgaben der VS.
	\end{itemize}
\end{itemize}



\newpage
\vspace*{0.5cm}
\thispagestyle{empty}

%\setcounter{section}{-1}
\section*{Gesetze, Satzungen und Ordnungen – viel Text für die Verfasste Studierendenschaft}

Im Tagesgeschäft müssen durch die Ausführenden meist sehr detaillierte oder ab und an auch stark improvisierte Lösungen gefunden werden. Bei der Gestaltung und weiteren Ausgestaltung der Grundlagen für diese eigentliche Arbeit ist es aber sicherlich lohnenswert prinzipiell und weitergehend zu denken. Dementsprechend zahlt es sich auch aus einen Blick auf das zu werfen, was schon vorgedacht wurde oder bereits konkret festgelegt ist. Die gesetzlichen Grundlagen, auf denen die Verfasste Studierendenschaft fußt, muss man sich dafür meist mit einigem Aufwand zusammensuchen\footnote{Außerdem sollen sollen die aktiven Studierenden, ja auch ihrer eigentlichen Hauptaufgabe – nämlich ihrem eigenen Studium – nachgehen.}. Aus diesem Grund wurde hier versucht, die wichtigsten Gesetzestexte für den Gebrauch in den Gremien und Arbeitsgruppen zusammenzustellen, so dass nicht jeder selbst diesen Aufwand treiben muss. Außerdem sind die Paragraphen für den Laien oft nicht einfach verständlich weshalb hier zusätzlich der Versuch unternommen wurde die wichtigsten Punkte zusammenzufassen und etwas zu erläutern.

Natürlich kann das geschriebene Recht in vielen Fällen „verbogen“ werden und man kann manchmal mehr und manchmal weniger weit vom genauen Wortlaut abweichen, vorhandene Spielräume ausreizen. Gleichzeitig bestehen aber viele deutliche Vorgaben des Gesetzgebers und davon abgesehen lohnt sich sicher der Versuch, die ursprünglichen Intentionen der Verfasser nachzuvollziehen. Irgendwas werden „die“ sich ja schon auch gedacht haben ;-) Neben der Nachhaltigkeit kann man durch gut gefasste, ausgestaltende Regeln auch auf eine größere Rechtssicherheit hoffen. Dabei sollte man aber auch nicht versuchen ständig in “worst case”-Szenarien zu denken, um möglichst alle Eventualitäten zu regeln\footnote{Dies lohnt sich bei den meisten Angelegenheiten schon allein deshalb nicht, weil erfahrungsgemäß im Voraus gar nicht an alle Möglichkeiten gedacht werden kann.}. Letztendlich wird es immer eine einzelne Person oder eine kleine Gruppe geben, die Dinge in die Tat umsetzt und dabei den eigenen, hoffentlich gesunden Menschenverstand nutzen muss.

Dieses Dossier enthält daher vor allem die für das Thema Verfasste Studierendenschaft und den Übergang in diese neue Form relevanten Abschnitte aus den entsprechenden Gesetzen. Daneben gibt es in der Art eines Glossars ein paar Erläuterungen, die beim Verständnis dieser Texte helfen können und den Autoren wichtig erscheinende Punkte aus den eher unübersichtlichen Gesetzestexten zusammenfassen. Allerdings ist nicht beabsichtigt alle Aspekte zu beleuchten, sondern mehr die Fragen zu notieren, die sich bisher stellten – inklusive der soweit gefundenen Antworten. Diese Erläuterungen sowie die Zusammenstellung der Texte wurde \textbf{von juristischen Laien erstellt} und auch wenn wir uns Mühe gegeben haben, kann natürlich nicht für Richtigkeit und schon gar nicht für Vollständigkeit garantiert werden.

Nach dem vorangestellten Abkürzungsverzeichnis und der „ToDo-Liste“, also zuerst ein Überblick über die drei relevanten Gesetze: dem \textbf{Verfasste-Studierendenschafts-Gesetz}, das seinerseits das \textbf{Gesetz über die Errichtung der Verfassten Studierendenschaft} enthält und dem \textbf{Landeshochschulgesetz (LHG)}. Als viertes Regelwerk ist für Ulm die schon abgestimmte \textbf{Organisationssatzung der Studierendenschaft} zu nennen, die bereits mit Erläuterungen veröffentlicht ist (Link unten). Darauf folgt zum schnellen Nachschlagen der \textbf{Glossar} mit Erläuterungen und abschließend die \textbf{Gesetzestexte im Wortlaut}.

\begin{flushright}
	\textit{Ulm, im Herbst 2013}
\end{flushright}



\newpage
\section{Zu den Gesetzestexten}
\subsection{Verfasste-Studierendenschafts-Gesetz (VerfStudG)}

Die insgesamt 12 Artikel dieses Gesetzes stellen zum Teil wiederum eigenständige Gesetze dar – wie z.B. das Gesetz über die Errichtung der Verfassten Studierendenschaft aus Artikel 3 – oder passen bereits bestehende Gesetze – wie z.B. das Landeshochschulgesetz – an, entsprechend der Absicht eine VS einzuführen.

\begin{itemize}
	\item Voller Titel: Gesetz zur Einführung einer Verfassten Studierendenschaft und zur Stärkung der akademischen Weiterbildung
	\item \sloppy \url{http://mwk.baden-wuerttemberg.de/fileadmin/pdf/hochschulen/Verfasste-Studierendenschaft/GBl-2012_457.pdf}
	\item Veröffentlicht im Gesetzesblatt BaWü Nr. 11/2012 vom 13. Juli 2012.
	\item Gesetzesentwurf (inkl. Zielsetzung, Begründung, Anhörungsergebnisse) in der Drucksache des Landtags 15 / 1600 vom 24.04.2012, siehe \url{http://www.landtag-bw.de/Dokumente}
\end{itemize}

Das Gesetz ist in mehrere Artikel gegliedert, zur Übersicht hier eine kurze Zusammenfassung (kein Anspruch auf Vollständigkeit/Korrektheit):

\begin{itemize}
	\item Artikel 1: \textbf{Errichtung einer Verfassten Studierendenschaft} \textrightarrow~ kurze grundsätzliche Erklärung und Definition des Rechtsstatus, s.u..
	\item Artikel 2: Änderung des Landeshochschulgesetzes \textrightarrow~ kleinere Anpassungen, redaktionelle Änderungen und v.a. Einfügen von drei neuen §§ (65, 65 a und 65 b) zur Verfassten Studierendenschaft, s.u..
	\begin{itemize}
		\item einige kleinere Änderungen, gerade auch mit Bezug zur akademischen Weiterbildung, lohnt sich evtl. zu lesen bei Interessen jenseits des Themas VS.
		\item Sonst noch:
		\begin{itemize}
			\item Allgemein für Studentische Belange interessant ist der hinzugekommene § 36 a: „Anerkennung von Studien- und Prüfungsleistungen sowie Studienabschlüssen.“
			\item § 63 bekommt einen neuen Absatz (3), der Minderjährige sozusagen für selbstständig handlungsfähig bei Verwaltungssachen in Studienangelegenheiten erklärt.
			\item In § 70 (1) wurde noch was zur Anerkennung / Akkreditierung ergänzt.
		\end{itemize}
	\end{itemize}
	\item Artikel 3: \textbf{Gesetz über die Errichtung der Verfassten Studierendenschaft}, s.u.
	\item Artikel 4: Änderung des Qualitätssicherungsgesetzes \textrightarrow~ „Über die Verwendung der Qualitätssicherungsmittel ist im Einvernehmen mit einer Vertretung der Studierenden zu entscheiden; …“, Details eben im Qualitätssicherungsgesetz: \sloppy \url{http://www.landesrecht-bw.de/jportal/portal/t/p98/page/bsbawueprod.psml?pid=Dokumentanzeige&showdoccase=1&js_peid=Trefferliste&documentnumber=1&numberofresults=1&fromdoctodoc=yes&doc.id=jlr-QualSiGBWrahmen&doc.part=X&doc.price=0.0#focuspoint}
	\item Artikel 5 – 8, 10 und 11: Sonstiges \textrightarrow~ selber nachschaun.
	\item Artikel 9: „Änderungen des Hochschulzulassungsgesetzes“
	\item Artikel 12: \textbf{„Inkrafttreten, Übergangsbestimmungen“}, s.u.
\end{itemize}

\subsection{Gesetz über die Errichtung der Verfassten Studierendenschaft}

\begin{itemize}
	\item Beschreibt die Absicht des Gesetzgebers, also der Landesregierung, dass eine VS eingerichtet werden soll und welche Rechtsform diese hat.
	\item Regelt das Procedere, wie diese Einführung von statten gehen soll: Vorschläge für die Organisationssatzung, Urabstimmung, erste Wahlen, …
	\item Regelt die erstmalige Konstituierung der VS und enthält eine Rückfallregelung für den besonderen Fall, in dem die VS nicht nach obigem Procedere vor dem 31.12.2013 zustande kommt.
	\item Auch das Zustandekommen der Landesweiten Studierendenvertretung wird geregelt.
\end{itemize}


\subsection{Landeshochschulgesetz (LHG)}
\begin{itemize}
	\item Voller Titel: Gesetz über die Hochschulen in Baden-Württemberg
	\item \sloppy  \url{http://www.landesrecht-bw.de/jportal/portal/t/1jxm/page/bsbawueprod.psml?pid=Dokumentanzeige&showdoccase=1&js_peid=Trefferliste&fromdoctodoc=yes&doc.id=jlr-HSchulGBWV16P65c&doc.part=X&doc.price=0.0#focuspoint}
	\item \textbf{… und dort eben ganz konkret die §§ 65, 65 a und 65 b, die für die Einführung der VS komplett neu gefasst oder eingeführt wurden.}
	\item Welche anderen Teile des LHG auch konkret auf die VS anzuwenden sind ist weitestgehend unklar. Evtl. lohnt es sich ein entsprechendes Gutachten mal landesweit organisiert durchzuführen oder in Auftrag zu geben.
	\item Grundlage für dieses Dossier ist die Version mit Gültigkeit vom 14.07.2012 bis zum 31.12.2013.
\end{itemize}

\subsection{Organisationssatzung (OS)}

Ist durch die Urabstimmung Anfang 2013 und die darauf folgende rechtliche Genehmigung, Unterzeichnung und Veröffentlichung durch den Universitätsvorstand (Präsident) das bisher einzige weitere Dokument, das die „Spielregeln“ für die Studierendenschaft der Uni Ulm festschreibt – neben den natürlich sowieso immer gültigen Gesetzen und darauf basierenden Vorschriften.

\begin{itemize}
	\item gültige Fassung: \sloppy  \url{http://www.uni-ulm.de/fileadmin/website_uni_ulm/stuve/verfasste_studierendenschaft/dokumente/Organisationssatzung.pdf}
	\item Satzungsvorschlag, der Erläuterungen enthält: \url{http://www.uni-ulm.de/fileadmin/website_uni_ulm/stuve/verfasste_studierendenschaft/dokumente/Organisationssatzung_erl%C3%A4utert.pdf}
	\item Neben diesen Erläuterungen gibt es die Hoffnung, bald noch eine kommentierte Version der Organisationssatzung vorlegen zu können.
	\item Des weiteren werden zumindest eine Beitrags-, eine  Wahl- und eine Finanzordnung erlassen werden müssen.
\end{itemize}


\newpage
\section{Glossar}
\label{sec:Glossar}

Alphabetisch sortiert.



\paragraph{Allgemeiner Studierendenausschuss (AStA)}

Im Allgemeinen, d.h. in den meisten deutschen Bundesländern, wird das exekutive und gleichzeitig meist auch repräsentative Organ der Studierendenschaft als AStA bezeichnet. Die Ulmer StEx ist mit dem AStA gleichzusetzen. Dieser alternative Begriff wurde gebildet, um den exekutiven/operativen Aspekt dieses Organs deutlicher zu betonen und gleichzeitig vom Konzept des „alten AStA“ Abstand zu nehmen, das in Baden-Württemberg über 30 Jahre lang nur eine stark beschnittene offizielle studentische Selbstvertretung erlaubte.



\paragraph{Beiträge}

… werden hauptsächlich in LHG § 65 a (5) geregelt:
\begin{itemize}
	\item Regelung in einer Beitragsordnung, dort mindestens: Beitragspflicht, Höhe, Fälligkeit. Diese Ordnung hat Satzungsrang, muss also vom Hochschulvorstand genehmigt werden (siehe auch Rechtsaufsicht).
	\item Müssen „angemessen“ sein, die sozialen Belange der Studierenden sind zu berücksichtigen.
	\item Werden von der Hochschule unentgeltlich eingezogen.
\end{itemize}
Beiträge sind nicht dasselbe wie Gebühren! Beiträge werden dafür bezahlt, dass entsprechende Leistungen in Anspruch genommen werden können, unabhängig davon, ob dies tatsächlich passiert. Gebühren werden für eine konkret in Anspruch genommene Leistung erhoben. Siehe Wikipedia: \url{https://de.wikipedia.org/wiki/Beitrag}.



\paragraph{nicht-Benachteiligung}\label{Glossar: nicht-Benachteiligung}

… ist in LHG § 9 (7) Satz 2 mit deutlichen Worten geregelt: „Die Mitglieder dürfen wegen ihrer Tätigkeit in der Selbstverwaltung nicht benachteiligt werden.“ Die jeweils konkrete Umsetzung dürfte aber nicht einfach oder eindeutig sein. Sie gilt erstmal für Mitarbeit in den Gremien der akademischen Selbstverwaltung. Für die Mitarbeit in den Organen der Studierendenschaft wird dann in LHG § 65 (7) auch auf genau diesen Satz verwiesen. Das Verbot von Benachteiligung gilt also auch für die Mitgliedschaft in den Organen der VS. Außerdem wird dort auch auf LHG § 32 (6) verwiesen: „Eine Tätigkeit als gewähltes Mitglied in gesetzlich vorgesehenen Gremien oder satzungsmäßigen Organen der Hochschule oder des Studierendenwerks während mindestens eines Jahres kann bei der Berechnung der Prüfungsfristen bis zu einem Studienjahr unberücksichtigt bleiben; die Entscheidung darüber trifft die Rektorin oder der Rektor.“

Spannend ist also die Frage, wie umgesetzt wird, dass keinem sich in den Gremien engagierenden Mitglieder daraus ein Nachteil entsteht. Für Einzelfälle könnte man ableiten, dass z.B. immer ein Ersatztermin oder eine andere vertretbare Möglichkeit zum Ersatz angeboten werden muss, wenn eine Pflichtveranstaltung mit einer Gremiensitzung kollidiert oder dass man auf Grund von zu vielen Fehlterminen eben nicht Durchfallen darf. Je nachdem wie sehr man sich streiten will, kann man sich auf den Standpunkt stellen, dass die Uni sich sogar um die Organisation der Ersatzmöglichkeit kümmern muss. Erstmal sollte man sich aber sicher immer einfach direkt mit dem zuständigen Dozenten in Verbindung setzen und gemeinsam nach einer Lösung suchen und dabei notfalls auf diesen Rechtsanspruch hinweisen. (Noch weiter gesponnen: auch wenn es sich nicht um einen Pflichttermin handelt, man aber begründen kann, dass ein Versäumen des Termins von Nachteil ist, „dürfte“ das auch nicht sein.)

Ob solche evtl. ständigen Einzelfallösungen praktikabel sind, ist sicher fraglich – z.B. gerade bei den vielen Pflichtveranstaltungen im Medizinstudium mit wenigen Fehlterminen. Darum gibt's wohl auch LHG § 32 (6), der vorsieht, dass man Fristen aus den Prüfungsordnungen erhöhen kann. Man kann vermuten, dass das als pauschaler Ausgleich gedacht ist.

Siehe auch \nameref{Glossar: Ehrenamt}.



\paragraph{Beschäftigte}\label{Glossar: Beschäftigte}

… der Studierendenschaft unterliegen einer Bindung an den selben Tarif wie die Beschäftigten der Hochschule \textit{(LHG § 65 b (1))}. D.h. der „Tarifvertrag für den öffentlichen Dienst der Länder“ (TV-L) ist anzuwenden, einschließlich seiner Regelungen zur Eingruppierung.

Ab einer gewissen Anzahl von Beschäftigten haben diese selbstverständlich auch das Recht sich gewerkschaftlich zu organisieren und z.B. einen Personalrat zu gründen. Das ist aber der Erfahrung an anderen Hochschulen nach nicht besonders problematisch. Der Personalrat der Hochschule ist nicht zuständig.

Siehe auch \nameref{Glossar: Ehrenamt}.



\paragraph{Ehrenamt}\label{Glossar: Ehrenamt}

Alle Mitglieder der Organe arbeiten laut LHG § 65 a (7) ehrenamtlich, die entsprechenden ersten beiden Sätze lauten: „Die Mitglieder in den Organen der Studierendenschaft üben ihre Tätigkeit ehrenamtlich aus. Das legislative Organ kann eine angemessene Aufwandsentschädigung festsetzen.“ Dies trifft laut der Definition in § 3 der Ulmer OS auch auf die StEx zu. Die Referentinnen hingegen sind in dieser Funktion z.B.~keine Organmitglieder, da sie nicht in OS § 3 aufgezählt sind. Siehe hierzu also auch \nameref{Glossar: Beschäftigte}.

Wie dieses Ehrenamt mit dem Studium vereinbar ist sagt dann LHG § 65 a (7) Satz 3, siehe dazu \nameref{Glossar: nicht-Benachteiligung}.

Offene Frage: Wie passte eine „angemessene Aufwandsentschädigung“, also deren Höhe, zum geforderten Ehrenamt? In Ulm wurde in der bisherigen Diskussion für die sieben Mitglieder der StudierendenExekutive eine monatliche Aufwandsentschädigung von 400 oder 600 € pro Monat veranschlagt, da sie neben der Organtätigkeit auch die konkrete Geschäftsführung inne haben und der große Umfang und Anspruch dieser Aufgaben zu berücksichtigen ist. Leider wurde noch keine Option gefunden, eine Aufwandsentschädigung in diesem Umfang rechtlich sauber (Steuer, Arbeitsrecht, …) und praktikabel zu leisten. Unter den Stichworten Übungsleiterpauschale, Steuerfreibetrag für ehrenamtliche Aufwandsentschädigungen etc.~scheint es nicht möglich zu sein. Vielleicht lohnt sich hier auch ein Blick darauf wie die Entlohnung von Bürgermeistern kleinerer Gemeinden oder das Sitzungsgeld z.B.~für Kirchengemeinderäte gehandhabt wird.



\paragraph{Fachschaft, FachbereichsVertretung, FS}\label{Glossar: Fachschaft, Fachbereichsvertretung, FS}

Der Begriff „Fachschaft“ ist per Gesetz als Gesamtheit aller Studierenden einer Fakultät definiert, es gibt an der Uni Ulm also offiziell nur 4(!) Fachschaften.
Diese grobe Aufteilung hat sich in Ulm aber schon lange als nicht passend erwiesen (Subsidiaritätsprinzip) und gelebt wird deshalb eine Organisation in mehr als 10 „Fachschaften“ (umgangssprachlich), die sich um die Studierenden in sinngemäß zusammengehörenden Fächern kümmern.
Dementsprechend legt die OS die sogenannten „Fachbereichsvertretung“, abgekürzt FSen, fest und ordnet ihnen bestimmte Fächer bzw. deren Studierende zu.

Dass eine solche feingliedrigere Aufteilung angemessen sein kann findet auch der Gesetzgeber in der Gesetzesbegründung \textit{(Landtag BW Drucksache 15 / 1600, Seite 36)}.
Teilweise problematisch ist, aber dass die Studierenden vor Ort seit jeher der Begriff „Fachschaft“ für die genannten kleineren Organisationseinheiten verwenden, die legale Definition aber eine andere ist.

\textbf{\emph{Achtung!}} Da es nun leider diese begriffliche Diskrepanz gibt muss man v.a. wenn es um Offizielles geht auf einen korrekten Gebrauch geachtet werden! Beispielsweise \textbf{muss in den Protokollen der FSen immer der richtige Begriff Fachbereichsvertretung verwendet werden}.

Siehe hierzu auch \nameref{Glossar: Wahlen}: das Verbot der Bildung von Wahlkreisen (entsprechend allen Studierenden einer Fachbereichsvertretung) und der Wahl in Vollversammlungen trifft auf die Ulmer Fachbereichsvertretungen vermutlich nicht zu, sofern es sich um Wahlen bezüglich der Angelegenheiten der Fachbereichsvertretungen handelt.



\paragraph{Finanzen, Haushalt, Wirtschaftsführung\label{Glossar:Finanzen, Haushalt, Wirtschaftsführung}}

Fast der ganze § 65 b des LHG dreht sich um das liebe Geld und dessen Verwaltung. In gewisser Kürze:
\begin{itemize}
	\item Vorschriften des Landes Baden-Württemberg sind zu beachten, insbesondere die Landeshaushaltsordnung (LHO).
	\item Um mit dem Geld arbeiten zu können und eine Maßgabe für die Ausgaben sowie Grundlage bzgl. der Rechenschaft zu haben muss ein Wirtschafts- oder Haushaltsplan geführt werden. Welche Form es sein soll, muss festgeschrieben oder entschieden werden.
\end{itemize}

Folgende Akteure und Maßnahmen sind für die Finanzverwaltung und -aufsicht vorgesehen:
\begin{itemize}
	\item \textbf{Finanzreferent der Studierendenschaft}: für die Wirtschaftsverwaltung der Studierendenschaft hauptsächlich verantwortlich; „arbeitet mit dem Beauftragten für den Haushalt zusammen“ \textit{(LHG § 65 b (2))}.
	\item \textbf{Beauftragter für den Haushalt} (auch hauptamtlicher Finanzer): „Befähigung für den gehobenen Verwaltungsdienst hat oder in vergleichbarer Weise über nachgewiesene Fachkenntnisse im Haushaltsrecht verfügt.“; unmittelbar dem Vorsitzenden unterstellt; kann Studierender oder im Hauptberuf im Dienst der Uni sein. Mehr dazu in LHG im entsprechenden § 65 b (2). Der Gesetzesentwurf erläutert außerdem: „Der Beauftragte für den Haushalt kann entsprechend [LHG] § 16 Absatz 2 Satz 5 \textbf{einer Maßnahme widersprechen}, wenn er sie für rechtswidrig oder mit den Grundsätzen der Wirtschaftlichkeit nicht für vertretbar hält. In diesem Fall ist vom Vorsitzenden des exekutiven Organs der Studierendenschaft eine Entscheidung des legislativen Organs herbeizuführen, welches abschließend über den Vollzug der Maßnahme entscheidet.“ \textit{(Landtag BW Drucksache 15 / 1600, Seite 38f)}.
	\item \textbf{Rechnungsprüfung}: durch eine Fachkundige Person mit der Befähigung für den gehobenen Verwaltungsdienst (ungleich des Beauftragten für den Haushalt) oder durch die Verwaltung der Hochschule (wenn diese das möchte).
	\item \textbf{Landesrechnungshof}: hat die „Haushalts- und Wirtschaftsführung der Studierendenschaft“ \textit{(LHG § 65 b (3))} zu prüfen, neben der o.g.~Rechnungsprüfung.
\end{itemize}

Sonst noch:
\begin{itemize}
	\item Der Vorstand der Hochschule entlastet.
	\item LHG § 65 b (4): „Für Verbindlichkeiten haftet die Studierendenschaft mit ihrem Vermögen. Die Hochschule und das Land haften nicht für Verbindlichkeiten der Studierendenschaft.“
	\item Darlehen dürfen nicht Aufgenommen oder Vergeben werden; Beteiligung an oder Gründung von Unternehmen ist nur mit Zustimmung des Vorstands der Hochschule möglich. Siehe (7).
\end{itemize}

Bei diesem Thema lohnt es sich sicher noch einiges an Expertise – z.B. von Steuerberatern oder dem Finanzamt – einzuholen.



\paragraph{Gliedkörperschaft (der Hochschule)}

Als solche ist die VS in LHG § 65 und VerfStudG Art. 1 definiert; außerdem als „rechtsfähige Körperschaft des öffentlichen Rechts“, siehe also auch unter diesen Begriffen.

Es ist nicht klar, was der Gesetzgeber mit der Definition als „Gliedkörperschaft der Hochschule“ bezwecken wollte, denn es lassen sich direkt keine praktischen Konsequenzen zur Definition des rechtlichen Verhältnisses zwischen Hochschule und VS ableiten. Eine mögliche Umschreibung könnte man so fassen: „man gehört dazu, aber eben nicht so ganz, oder vielleicht auch umgekehrt“. Es bleibt also wohl nichts anderes übrig, als sich auf konkrete genannte Punkte (Rechtsaufsicht, Überlassung von Räumen etc.) zu beziehen und das Verhältnis HS ./.~VS mit der Zeit vor Ort auszudefinieren. Evtl. gibt es eine steuerrechtliche Relevanz, dazu wurde bereits eine Anfrage beim Finanzamt angestoßen.

Die Gesetzesbegründung lässt evtl. auf die Intention hinter der Formulierung schließen: „Die Qualifizierung als Gliedkörperschaft bringt zum Ausdruck, dass sie in der Trägerschaft der jeweiligen Hochschule – und nicht unmittelbar des Landes – steht. Sie unterliegt Bindungen gegenüber ihrer Hochschule jedoch nur insoweit, als dieses Gesetz solche vorsieht. So ordnet das Gesetz beispielsweise Genehmigungs- und Zustimmungsvorbehalte der jeweiligen Hochschule zu; ferner wird der Hochschule auch die Rechtsaufsicht über ihre Studierendenschaft übertragen.“ \textit{(Landtag BW Drucksache 15 / 1600, Seite 32)}.

Der Begriff Gliedkörperschaft ist damit nicht positiv definiert, sondern es gibt nur die Abgrenzung, dass der Träger nicht das Land sondern die Hochschule ist. Gliedkörperschaft ist wohl synonym zu Teilkörperschaft, das Synonym bringt aber auch keine klarere Definition oder einen Zweck der Fomulierung jenseits der Abgrenzung mit sich. Der Begriff Gliedkörperschaft oder Teilkörperschaft wird sonst wohl nur für Unikliniken und medizinische Fakultäten genutzt. Diese Referenz passt aber nicht ganz, es gibt dabei etwas andere Formulierungen. Eine Referenz auf diesen Aspekt findet sich in \emph{Das Hamburger Modell der Gliedkörperschaft}, Prof. Dr. Dr. U. Koch-Gromus, Dekan der Medizinischen Fakultät der Universität Hamburg, \url{http://www.mft-online.de/files/seite_107.pdf}.

Auch der Rechtsabteilung der Uni Ulm ist ein wirklicher Sinn hinter der Verwendung dieses Begriffes bzw. welche praktischen Konsequenzen daraus im Bezug auf Verhältnis von Hochschule und VS abzuleiten sind nicht klar.



\paragraph{Konstituierung}

Unter Konstituierung versteht man das (erstmalige) Zusammenkommen eines neuen Gremiums und damit den jeweils ersten Arbeitsschritt eines Gremiums. Dieser wird dadurch abgeschlossen, dass das Gremium arbeitsfähig wird (z.B.~indem es einen Vorsitzenden wählt und dieser die Leitung übernimmt). Die Konstituierung wird meist „von außen“ angestoßen (bspw.~ durch ein anderes Organ); wurde ein Gremium korrekt gewählt können sich seine Mitglieder aber auch selbst organisieren um sich zu konstituieren.

Zur erstmaligen Konstituierung der VS: Die neue StuVe, also die Uni Ulmer VS, ist dann vollständig konstituiert, „wenn sich das letzte Organ auf zentraler Ebene der Studierendenschaft konstituiert hat.“ Für Ulm ist dieses letzte Organ die StEx, die sich nach StuPa und FSR konstituieren muss. (Gesetz über die Errichtung der Verfassten Studierendenschaft § 1 (5)). Dies musste bis zum 31.12.2013 geschehen sein, da sonst die Regelungen für diesen „besonderen Fall“ in Kraft treten würden (VerfStudG Art. 3 §§ 2 und 3). Siehe hierzu auch \emph{Übergang zur VS}.



\paragraph{Körperschaft des öffentlichen Rechts (KdöR)}

\textit{\%TODO}

Siehe Wikipedia: \sloppy \url{https://de.wikipedia.org/wiki/K%C3%B6rperschaft_des_%C3%B6ffentlichen_Rechts_%28Deutschland%29}



\paragraph{Landesweite Studierendenvertretung}

… auch LaStuVe. Sie muss nach der Konstituierung aller VSen der Hochschulen gebildet werden \textit{(Gesetz über die Errichtung der Verfassten Studierendenschaft § 4)}. LHG § 65 a (8) regelt weiteres, v.a.~muss für diese auch eine GO abgestimmt werden, die z.B. regelt, wie die Landesebene durch die Studierendenschaften finanziert wird.



\paragraph{Öffentliches Recht}

\textit{\%TODO}

Siehe Wikipedia: \url{https://de.wikipedia.org/wiki/%C3%96ffentliches_Recht}



\paragraph{Politisches Mandat}

Es wurde immer wieder diskutiert, ob die VS und deren Organe, Vorsitzende oder Sprecher nun ein hochschulpolitisches oder allgemeinpolitisches Mandat bekommen sollen.
Mit einem allgemeinpolitischen Mandat wäre es der VS erlaubt sich zu jedem beliebigen Thema politisch zu positionieren, zu äußern etc.
Im Gesetz steht nun: „Im Rahmen der Erfüllung ihrer Aufgaben nimmt die Studierendenschaft ein politisches Mandat wahr. Sie wahrt nach den verfassungsrechtlichen Grundsätzen die weltanschauliche, religiöse und parteipolitische Neutralität.“ \textit{(LHG § 65 (4))}.
Diese diskutierten Begriffe tauchen also nicht auf, es ist stattdessen die Rede von einem „politische Mandat“, das die VS im „Rahmen der Erfüllung ihrer Aufgaben“ wahrnehmen soll. 

Schaut man sich die Breite und den allgemeinen Anspruch der Aufgaben der VS an, z.B.~im Bezug auf die politische Bildung der Studierenden und Förderung deren Verantwortungsbewusstseins, kann dieses Mandat vermutlich sehr weit ausgedehnt werden. \textbf{Jedoch ist Vorsicht geboten: es muss immer ein Bezug zu Studierenden, Studium, Hoschule oder dergleichen vorhanden sein.}

In der Begründung zum Gesetzesentwurf findet sich diese Einschränkung deutlicher als im Wortlaut der §§-en: „Die im Gesetz vorgesehenen Aufgaben geben der Studierendenschaft die Möglichkeit, sich umfassend für die Belange der Studierenden einzusetzen und zu hochschulpolitischen Themen Stellung zu nehmen. Die Aufgaben berücksichtigen die verfassungsrechtlichen Grenzen, die sich aus der Pflichtmitgliedschaft aller Studierenden in der Studierendenschaft ergeben, und begründen kein sogenanntes »allgemeinpolitisches Mandat«.“ \textit{(Landtag BW Drucksache 15 / 1600, Seite 24)}. Und an anderer Stelle: „[…] beinhaltet nicht die Befugnis, im Namen der Studierendenschaft eigene politische Forderungen zu formulieren und zu begründen, die über die oben genannten Belange der Studierenden hinausgehen.“ Etwas später dann: „Die Studierendenschaft erhält jedoch keine Befugnis, zu allgemeinpolitischen Themen Stellung zu nehmen, die nur am Rande mit wissenschaftlichen Erkenntnissen im Zusammenhang stehen oder die nicht die Gruppe der Studierenden betreffen.“ Die Landesregierung reitet in der Begründung geradezu auf diesem Aspekt herum, denn etwas später heißt es dann nochmal: „Der Studierendenschaft wird ein begrenztes partikuläres politisches Mandat eingeräumt, welches auf die Erfüllung der in Absatz 2 genannten Aufgaben und auf die Wahrnehmung gruppenspezifischer studentischer Belange beschränkt ist. Das Mandat rechtfertigt sich aus dem Auftrag der Studierendenschaft, sich bei hochschul- und studienspezifischen Belangen gegenüber der Hochschule und der Politik zu positionieren, Anregungen und Kritik vorzubringen und für ihre Auffassung zu werben, so, wie dies auch anderen Körperschaften des öffentlichen Rechts im Rahmen ihrer Aufgaben möglich ist. Die Regelung begründet kein allgemeinpolitisches Mandat der Studierendenschaft. Das Neutralitätsgebot in Satz 2 bedeutet keine Einschränkung des Mandats der Studierendenschaft, sondern stellt klar, dass die Studierendenschaft als öffentlich-rechtliche Körperschaft das sich aus Verfassungsrecht ergebende Gebot zur Wahrung der weltanschaulichen, religiösen und parteipolitischen Neutralität zu beachten hat. Die Studierendenschaft wird durch dieses Gebot nicht daran gehindert, sich bei Themen im Rahmen ihrer Aufgaben zu positionieren.“ \textit{(alle vorigen Zitate in Landtag BW Drucksache 15 / 1600, Seite 33 f)}.



\paragraph{Rechtsaufsicht}

… über die Studierendenschaft hat das Rektorat Hochschule. U.a. heißt das ganz konkret, dass Satzungen und Haushaltsplan genehmigt werden müssen, nicht-Genehmigung aber nur bei Rechtswidrigkeit \textit{(LHG § 65 b (6))}. Also sollte man bei diesen Angelegenheiten auch genügend Zeit für die rechtliche Prüfung durch die Hochschule einplanen und sich am besten möglichst früh mit der Rechtsabteilung abstimmen.

Die Rechtsaufsicht geht aber auch weiter \textit{(LHG §§ 67 (1) und 68 (1), (3) und (4))}. Die Universität \emph{kann} Berichte und Auskünfte verlangen, wenn diese dem Zweck dienen, ein Handeln auf seine Rechtmäßigkeit zu überprüfen. Sie \emph{kann} rechtswidrige Beschlüsse beanstanden und im Extremfall sogar Anordnungen und Maßnahmen an Stelle der VS treffen. Im Konfliktfall kann das Wissenschaftsministerium eingeschaltet werden \textit{(Landtag BW Drucksache 15 / 1600, Seite 39f)}.



\paragraph{Rechtsfähigkeit}

Die VS kann (als Organisation oder eben Körperschaft) direkt Träger von Rechten und Pflichten sein. Sie kann z.B. selbst und in eigenem Namen Verträge abschließen und Personal einstellen. Ebenso haftet sie für ihr Tun aber auch selbst. Das war beim „alten AStA“ anders – Träger von Rechten und Pflichten war jeweils die Universität.



\paragraph{Satzung}

Die Satzung sind die Regeln, die sich eine VS im Sinne der Satzungsautonomie selbst geben kann. Neben „Satzung“ und hier insbesondere „Organisationssatzung“ taucht dann auch immer der Begriff „Ordnung“ auf. An diesen Bezeichnungen lässt sich der Status der Regeln jedoch nicht festmachen: eine „Ordnung“ kann durchaus \emph{Satzungscharakter} haben, d.h. sie enthält Regelungen, die alle Mitglieder rechtlich binden, quasi wie ein Gesetz. Dies trifft in der Regel für die Ordnungen der Universität zu. Im Gegensatz dazu gibt es dann z.B. noch Richtlinien oder Vorschriften, die interne Handlungsanweisungen geben, nach außen aber nur bedingt Rechtswirksamkeit entfalten (bspw.~Geschäftsordnungen, Dienstanweisungen).

Eine Ordnung bzw. Satzung muss immer von einem satzungsgebenden Organ erlassen werden und dazu muss es gesetzlich ermächtigt sein. In der VS wird dieses Organ im LHG entsprechend als „legislatives Organ“ bezeichnet, dessen Rolle in Ulm das StuPa inne hat.

%TODO Die \emph{Rechtsfolge} macht den Unterschied...

Konkret wird im Gesetz zuerst einmal die Organisationssatzung (OS) genannt und dabei definiert, was die VS als \textbf{Grundlagen für ihre „Organisation“} festlegen muss: Definition der Organe (Zusammensetzung, Zuständigkeit), Beschlussfassung, Bekanntgabe der Beschlüsse, Wahlgrundsätze \textit{(LHG § 65 a (1))}. In LHG § 65 a (5) wird dann z.B. auch Beitragsordnung (BO) genannt, die „als Satzung erlassen“ wird. Also zumindest die Organisationssatzung, Wahlordnung, Finanzordnung und Beitragsordnung (OS, WO, FO und BO) müssen vom StuPa beschlossen und von der \emph{Rechtsaufsicht} genehmigt werden.

Die Verfahrensordnung der Hochschule gilt übrigens nicht für die VS.



\paragraph{Schlichtungskommission}

Muss existieren, um einzelnen Studierenden zu ermöglichen sich über die Überschreitung der definierten Aufgaben der Studierendenschaft zu beschweren \textit{(LHG § 65 a (9))}; soll vor dem Gang vor ein Gericht angerufen werden, der „Rechtsweg“ wird durch anrufen der Schlichtungskommission jedoch nicht direkt berührt, z.B.~ was Fristen angeht.



\paragraph{Übergang zur VS}

„Bis zum Eingang der ersten von der Studierendenschaft erhobenen Beiträge stellt die Hochschule die Finanzierung, Personal- und Sachausstattung der Studierendenschaft im bisherigen, vor Inkrafttreten dieses Gesetzes an die Studierendenvertretung geleisteten Umfang sicher.“ (VerfStudG Art. 12 (3)) Die spannende Frage dabei ist was „Eingang“ des Geldes bedeutet: auf dem Konto der VS? Oder bei der einziehenden Hochschule?

In Ulm wurde z.B. gleich durch einen § in der urabgestimmten OS dafür gesorgt, dass die Höhe der \emph{Beiträge} und entsprechend weiteres festgelegt ist, wodurch von der Univerwaltung direkt Geld eingezogen werden konnte, noch bevor sich überhaupt ein allererstes neues Organ konstituiert hatte. Allerdings beschleunigte das natürlich nicht die vollständige Konstituierung, um dann voll geschäftsfähig Handeln zu können und nicht zuletzt muss man danach ja auch noch ein Konto aufmachen, auf das das Geld dann eingeht. Gleichzeitig mit dem Zeitpunkt der vollständigen Konstituierung wird der alte AStA aufgelöst: die o.g. Sicherstellung der bisherigen Leistungen durch die Hochschule hat dann kein organisatorisches Dach mehr und wird damit evtl. auch schwierig bzw. muss für den Übergang neu verhandelt werden.

Siehe auch \emph{Konstituierung der VS}.



\paragraph{Verfasste Studierendenschaft}

… abgekürzt VS und oft auch nur Studierendenschaft. Alle Studierenden (einer Universität) werden als „Studierendenschaft“ bezeichnet. „Verfasst“ ist diese durch die Verankerung im LHG als Körperschaft des öffentlichen Rechts. Damit besitzt die VS z.B. Rechtsfähigkeit, ein politisches Mandat oder Satzungs- und Finanzautonomie.

Die Organisation der VS, wird hauptsächlich in LHG § 65 a (3) vorgegeben:
\begin{itemize}
	\item „Die Organisation der Studierendenschaft muss wesentlichen demokratischen Grundsätzen entsprechen.“
	\item \emph{legislatives Organ} \textit{\%TODO}
	\item \emph{exekutives Organ} \textit{\%TODO}
	\item \emph{Zentrale Ebene} bedeutet hierbei hochschulweite oder gesamtuniversitäre Ebene, beispiele sind StuPa, FSR und StEx oder Senat und Präsidium. Im Gegensatz dazu sind die Fakultäten oder Fächer, also die FSen oder die Fakultätsräte und Studienkommissionen, nicht gemeint.
	\item \emph{Kollegialorgan}: die Mitglieder des Organs nehmen die entsprechenden Aufgaben gemeinsam Wahr, im Ggs. zu einem „Einzelorgan“ in dem nur Person die Arbeit macht.
\end{itemize}



\paragraph{Vorsitz}

Die Vorsitzende des exekutiven Organs „[vertritt] die Studierendenschaft im Außenverhältnis […]“ \textit{(Landtag BW Drucksache 15 / 1600, Seite 35)}. Sie kann somit auch als „Vorsitzende der Studierendenschaft“ bezeichnet werden, da sie für diese und somit im Namen aller Studierenden, „sprechen“ darf, ist also ein wesentliches Element der Repräsentation Sie\footnote{wiederum generisches Femininum} hat auch sonst eine gewisse Sonderrolle, siehe z.B. \emph{Beauftragte/r für den Haushalt}, eine Stelle die der Vorsitzenden direkt unterstellt ist.

Es ist evtl. sinnvoll eine Stellvertretung zu wählen. Die §§ 2 und 3 des Gesetzes über die Errichtung der Verfassten Studierendenschaft, die die Konstituierung im besonderen Fall regeln, sehen z.B. einen gewählten Stellvertreter vor. Es ist auch möglich, zwei Vorsitzende zu wählen, die dann gemeinschaftlich vertreten. Dies muss aber in der OS vorgesehen sein, was in Ulm aktuell nicht der Fall ist.



\paragraph{Wahlen}\label{Glossar: Wahlen}

Es muss ordentlich und natürlich heutigen demokratischen Standards entsprechend gewählt werden. LHG § 9 (8) findet Anwendung, d.h. Wahlen finden in der Regel nach den Grundsätzen der Verhältniswahl statt, sie müssen demokratischen Grundsätzen entsprechen, also vor allem frei, gleich und geheim sein. Es muss eine Wahlordnung (WO) geben. Dazu gibt es für die VS speziell in LHG § 65 a (2) und (3) ein paar Vorgaben: die Wahlen sollen gleichzeitig mit denen für die studentischen Senatsmitglieder stattfinden, die Wahlperiode soll ein Jahr betragen. Die WO muss auch festlegen, für welche Organe wie viele Mitglieder zu wählen sind.

In bestimmten Fällen kann in der Studierendenschaft im Ggs. zum grundsätzlichen Verbot \textit{(LHG § 9 (8) Satz 3, vgl. auch Landtag BW Drucksache 15 / 1600, Seite 35)} auch \textbf{in Vollversammlungen gewählt werden}. Außerdem ist mit dem genannten Satz die \textbf{„Bildung von Wahlkreisen“ verboten}. D.h.~für Wahlen dürfen die Studierenden nur auf der Ebene der Fakultäten und nicht in kleineren darunter liegenden „Wahlkreisen“ zusammengefasst werden. Siehe hierzu auch \nameref{Glossar: Fachschaft, FachbereichsVertretung, FS}.



\paragraph{Zuständigkeiten\label{Glossar:Zuständigkeiten}}

„[Die Studierendenschaft] hat unbeschadet der Zuständigkeit der Hochschule und des Studentenwerks die folgenden Aufgaben […]“ \textit{(LHG § 65 (2))}.

Diese Einleitung soll vermutlich ausdrücken, dass finanzielle Investitionen der Hochschule – etwa für langfristig gebundenes Personal, Leasing- oder Mietverträge etc. – „geschützt“ sind, also durch Aktivitäten der VS nicht geschädigt (also z.B.~untergraben) werden dürfen. Evtl.~ist daraus auch abzuleiten, dass es nicht zu viel Kompetenzgerangel geben soll. Dazu passend regelt LHG § 65 (5) dann auch, dass die Hochschule selbst sowie das Studentenwerk ganz bestimmte „Vorrechte“ oder „Vorrang“ haben: es wird definiert, wie man sich abstimmen muss, wenn die Studierendenschaft Aufgaben übernehmen möchte, die eigentlich zum Studentenwerk gehören oder wenn sie Sportangebote im großen Stil anbieten möchte.

\emph{Siehe auch \nameref{Glossar: Aufgaben}.}



\paragraph{Zwangsmitgliedschaft}
Der Gesetzgeber fasst im Gesetzesentwurf die Abwägung wie folgt zusammen: „Im Vorfeld wurde diskutiert, ob ein Recht zum Austritt aus der Verfassten Studierendenschaft vorgesehen werden sollte. Dafür spricht die Gewährleistung größerer Wahlfreiheit der Studierenden. Andererseits ist zu berücksichtigen, dass eine Verfasste Studierendenschaft ihre umfangreichen Aufgaben nur bewältigen kann, wenn sie über ausreichende finanzielle Mittel verfügt, die von allen Studierenden geleistet werden.

Letztlich entscheidet sich das Gesetz gegen ein Austrittsrecht, weil es für die Verfasste Studierendenschaft essentiell ist, für sich in Anspruch nehmen zu können, die Studierenden einer Hochschule insgesamt zu vertreten und unterschiedliche
Meinungsströmungen zu repräsentieren.“ \textit{(Landtag BW Drucksache 15 / 1600, Seite 3)}.




\appendix
\newpage

\section{Verfasste-Studierendenschafts-Gesetz – VerfStudG}

\emph{In Kraft getreten mit der Verkündigung am 13. Juli 2013, siehe dazu außerdem Artikel 12. Hier nur die noch relevanten Artikel.}

\subsection{Artikel 1 – Errichtung einer Verfassten Studierendenschaft}

An den Hochschulen des Landes Baden-Württemberg im Sinne des § 1 Absatz 2 des Landeshochschulgesetzes wird eine Verfasste Studierendenschaft 
nach Maßgabe dieses Gesetzes eingerichtet. Die Verfasste Studierendenschaft (Studierendenschaft) ist eine rechtsfähige Körperschaft des öffentlichen Rechts und als solche eine Gliedkörperschaft der Hochschule.

\bigskip
\emph{[Ausgelassen: Artikel 2 – Änderung des Landeshochschulgesetzes.]}


\subsection{Artikel 3 – Gesetz über die Errichtung der Verfassten Studierendenschaft}

\emph{[Ausgelassen: 
\begin{itemize}
	\item § 1 – Organisationssatzung, Abstimmung; Konstituierung im Regelfall
	\item § 2 – Konstituierung im besonderen Fall; Wahlen
	\item § 3 – Konstituierung im besonderen Fall; Organe]
\end{itemize}
}

\subsubsection*{§ 4 – Landesweite Vertretung der Studierendenschaft}

Nach Konstituierung aller Studierendenschaften des Landes Baden-Württemberg beruft der Vorsitzende des exekutiven Organs der Studierendenschaft der Hochschule mit der landesweit höchsten Zahl der immatrikulierten Studierenden die Vertreter der Studierendenschaften aller Hochschulen zur konstituierenden Sitzung ein; § 3 Absatz 1 Satz 3 gilt entsprechend. In der konstituierenden Sitzung beschließt die landesweite Vertretung der Studierendenschaft eine Geschäftsordnung nach § 65 a Absatz 8 Satz 2 LHG in der konstituierenden Sitzung beschließt die landesweite Vertretung der Studierendenschaft eine Geschäftsordnung nach § 65 a Absatz 8 Satz 2 LHG in der Fassung des Artikels 2 dieses Gesetzes.


\emph{[Ausgelassen: § 5 – Fachhochschulen für den öffentlichen Dienst]}


\bigskip
\emph{[Ausgelassen: Artikel 4 – 12, die sonstiges enthalten.]}




\newpage
\section{LandesHochschulGesetz: §§ zur Studierendenschaft}

\emph{Hier nur die §§ 65, 65 a und 65 b, die die Bestimmungen zur VS direkt enthalten, in der Fassung mit Gültigkeit ab dem 9.4.2014. Andere Stellen, die sich auch auf die VS beziehen sind nicht berücksichtigt.}


\subsection{§ 65 – Studierendenschaft}

(1) Die immatrikulierten Studierenden (Studierende) einer Hochschule bilden die Verfasste Studierendenschaft (Studierendenschaft). Sie ist eine rechtsfähige Körperschaft des öffentlichen Rechts und als solche eine Gliedkörperschaft der Hochschule.

(2) Die Studierendenschaft verwaltet ihre Angelegenheiten im Rahmen der gesetzlichen Bestimmungen selbst. Sie hat unbeschadet der Zuständigkeit der Hochschule und des
Studierendenwerks die folgenden Aufgaben:
\begin{enumerate}
	\item die Wahrnehmung der hochschulpolitischen, fachlichen und fachübergreifenden sowie der sozialen, wirtschaftlichen und kulturellen Belange der Studierenden,
	\item die Mitwirkung an den Aufgaben der Hochschulen nach den §§ 2 bis 7,
	\item die Förderung der politischen Bildung und des staatsbürgerlichen Verantwortungsbewusstseins der Studierenden,
	\item die Förderung der Chancengleichheit und den Abbau von Benachteiligungen innerhalb der Studierendenschaft,
	\item die Förderung der sportlichen Aktivitäten der Studierenden,
	\item die Pflege der überregionalen und internationalen Studierendenbeziehungen.
\end{enumerate}

(3) Zur Erfüllung ihrer Aufgaben ermöglicht die Studierendenschaft den Meinungsaustausch in der Gruppe der Studierenden und kann insbesondere auch zu solchen Fragen Stellung beziehen, die sich mit der gesellschaftlichen Aufgabenstellung der Hochschule, ihrem Beitrag zur nachhaltigen Entwicklung sowie mit der Anwendung der wissenschaftlichen Erkenntnisse und der Abschätzung ihrer Folgen für die Gesellschaft und die Natur beschäftigen.

(4) Im Rahmen der Erfüllung ihrer Aufgaben nimmt die Studierendenschaft ein politisches Mandat wahr. Sie wahrt nach den verfassungsrechtlichen Grundsätzen die weltanschauliche, religiöse und parteipolitische Neutralität.

(5) Beabsichtigt die Studierendenschaft, nicht nur vorübergehend konkrete Aufgaben oder Angebote innerhalb ihrer Zuständigkeit wahrzunehmen, die bereits von dem für die Hochschule zuständigen Studierendenwerk wahrgenommen werden, bedarf die Studierendenschaft für die Wahrnehmung der Aufgaben des Einvernehmens des Studierendenwerks. Beabsichtigt die Studierendenschaft, nicht nur vorübergehend die konkrete Wahrnehmung von Aufgaben und Angeboten innerhalb ihrer Zuständigkeit, die auch in den Aufgabenbereich des Studierendenwerks nach § 2 StWG fallen und von diesem derzeit nicht wahrgenommen werden, erfolgt die Aufgabenwahrnehmung im Benehmen mit dem zuständigen Studierendenwerk. Beabsichtigt die Studierendenschaft, nicht nur vorübergehend Sportaktivitäten anzubieten, die für sie mit erheblichen finanziellen Kosten verbunden sind, erfolgt dies im Einvernehmen mit der Hochschule.


\subsection{§ 65 a Organisation der Studierendenschaft; Beiträge}

(1) Die Studierendenschaft gibt sich eine Organisationssatzung; sie kann sich weitere Satzungen geben. Der Beschluss über die Organisationssatzung einschließlich ihrer Änderungen bedarf der Zustimmung von mindestens der Hälfte der an der Abstimmung teilnehmenden Studierenden. Die Organisationssatzung kann vorsehen, dass Änderungen der Organisationssatzung auch mit einer Mehrheit von zwei Dritteln der Mitglieder des legislativen Organs nach Absatz 3 Satz 2 beschlossen werden können. Die Satzungen der Studierendenschaft macht das Rektorat der Hochschule in der für Hochschulsatzungen vorgesehenen Weise als Satzungen der Gliedkörperschaft bekannt.

(2) Die Organisationssatzung legt die Zusammensetzung der Organe der Studierendenschaft und deren Zuständigkeit, die Beschlussfassung und die Bekanntgabe der Beschlüsse sowie die Grundsätze für die Wahlen fest, die frei, gleich, allgemein und geheim sind. Die Studierenden der Hochschule haben das aktive und passive Wahlrecht.

(3) Die Organisation der Studierendenschaft muss wesentlichen demokratischen Grundsätzen entsprechen. Die Organisationssatzung muss auf zentraler Ebene ein Kollegialorgan vorsehen, welches über die grundsätzlichen Angelegenheiten der Studierendenschaft einschließlich der sonstigen Satzungen beschließt (legislatives Organ); dieses Organ kann auch als Vollversammlung der Studierenden ausgestaltet sein. Die Organisationssatzung sieht ein exekutives Kollegialorgan vor, welches auch Teil des legislativen Organs sein kann; die Anzahl der Mitglieder des exekutiven Organs muss weniger als die Hälfte der Anzahl der Mitglieder des legislativen Organs betragen. Das exekutive Organ der Studierendenschaft hat eine oder einen Vorsitzenden, die oder der die Studierendenschaft vertritt. Die Organisationssatzung legt die Grundsätze für die Wahl der oder des Vorsitzenden fest und kann auch die Wahl von zwei Vorsitzenden vorsehen, welche die Studierendenschaft gemeinschaftlich vertreten. Sofern auf zentraler Ebene der Studierendenschaft keine unmittelbar von den Studierenden gewählten Vertreterinnen oder Vertreter handeln, ist die Legitimation dieser Vertreterinnen oder Vertreter aus anderen Organen der Hochschule oder der Studierendenschaft sicherzustellen, deren Mitglieder unmittelbar gewählt werden. Die Organisationssatzung kann vorsehen, dass die studentischen Senatsmitglieder dem legislativen Organ als stimmberechtigte Amtsmitglieder angehören; ferner soll sie vorsehen, dass die Wahlen zu den Vertreterinnen oder Vertretern der Studierendenschaft gleichzeitig mit der Wahl zu den studentischen Senatsmitgliedern stattfinden und die Wahlperiode ein Jahr beträgt; die Wahlen können sich auf mehrere Tage erstrecken.

(4) Die Studierenden einer Fakultät bilden eine Fachschaft, die eigene Organe wählen kann. Das Weitere regelt die Organisationssatzung der Studierendenschaft, die auch vorsehen kann, dass die jeweiligen studentischen Fakultätsratsmitglieder Organen der Fachschaft angehören. Die Organe der Fachschaft nehmen die fakultätsbezogenen Studienangelegenheiten und Aufgaben im Sinne des § 65 Absatz 2 auf Fakultätsebene wahr. An der DHBW wird eine Studierendenvertretung der örtlichen Studienakademie gebildet; das Weitere regelt die Organisationssatzung der Studierendenschaft der DHBW.

(5) Die Hochschule stellt der Studierendenschaft Räume unentgeltlich zur Verfügung. Für die Erfüllung ihrer Aufgaben erhebt die Studierendenschaft nach Maßgabe einer Beitragsordnung angemessene Beiträge von den Studierenden. In der Beitragsordnung sind die Beitragspflicht, die Beitragshöhe und die Fälligkeit der Beiträge zu regeln; die Beitragsordnung wird als Satzung erlassen. Bei der Festsetzung der Beitragshöhe sind die sozialen Belange der Studierenden zu berücksichtigen. Die Beiträge werden von der Hochschule unentgeltlich eingezogen.

(6) Die Organe der Studierendenschaft haben das Recht, im Rahmen ihrer Aufgaben Anträge an die zuständigen Kollegialorgane der Hochschule zu stellen; diese sind verpflichtet, sich mit den Anträgen zu befassen. Die Studierendenschaft kann nach Maßgabe ihrer Organisationssatzung jeweils eine Vertreterin oder einen Vertreter benennen, die oder der an allen Sitzungen des Senats und des Fakultätsrats mit beratender Stimme teilnehmen kann.

(7) Die Mitglieder in den Organen der Studierendenschaft üben ihre Tätigkeit ehrenamtlich aus. Das legislative Organ kann eine angemessene Aufwandsentschädigung festsetzen. Für die Tätigkeit in den Organen der Studierendenschaft gelten § 9 Absatz 7 Satz 2 und § 32 Absatz 6 entsprechend.

(8) Die Studierendenschaften der Hochschulen des Landes Baden-Württemberg bilden zur Wahrnehmung ihrer gemeinsamen Interessen eine landesweite Vertretung der Studierendenschaften. Näheres regelt eine Geschäftsordnung, die der Zustimmung von zwei Dritteln der Studierendenschaften aller Hochschulen bedarf. In der Geschäftsordnung wird auch die Finanzierung der landesweiten Vertretung durch die Studierendenschaften geregelt.

(9) Die Organisationssatzung der Studierendenschaft soll die Einrichtung einer Schlichtungskommission vorsehen. Die Schlichtungskommission kann von jeder oder jedem Studierenden der Hochschule mit der Behauptung angerufen werden, die Studierendenschaft habe in einem konkreten Einzelfall ihre Aufgaben nach § 65 Absätze 2 bis 4 überschritten. Einzelheiten der Schlichtungskommission einschließlich ihrer Besetzung regelt die Organisationssatzung der Studierendenschaft.


\subsection{§ 65 b Haushalt der Studierendenschaft; Aufsicht}

(1) Für die Haushalts- und Wirtschaftsführung sind die für das Land Baden-Württemberg geltenden Vorschriften, insbesondere die §§ 105 bis 111 LHO, entsprechend anzuwenden; die Aufgabe des zuständigen Ministeriums und des Finanz- und Wirtschaftsministeriums im Sinne der §§ 105 bis 111 LHO übernimmt das Rektorat der Hochschule. Die Organisationssatzung legt fest, wer die Entscheidung über die Führung eines Wirtschaftsplans (§ 110 LHO) anstelle eines Haushaltsplans (§ 106 LHO) trifft. Die Beschäftigten der Studierendenschaft unterliegen derselben Tarifbindung wie Beschäftigte der Hochschule.

(2) Das exekutive Kollegialorgan nach § 65 a Absatz 3 Satz 3 bestellt eine Beauftragte oder einen Beauftragten für den Haushalt im Sinne des § 9 LHO, die oder der die Befähigung für den gehobenen Verwaltungsdienst hat oder in vergleichbarer Weise über nachgewiesene Fachkenntnisse im Haushaltsrecht verfügt. Dienststelle der oder des Beauftragten für den Haushalt im Sinne des § 9 Absatz 1 Satz 1 LHO ist die Gliedkörperschaft. Sie oder er ist unmittelbar der oder dem Vorsitzenden des exekutiven Organs nach § 65 a Absatz 3 Satz 4 unterstellt; die oder der Vorsitzende gilt als Leiterin oder Leiter der Dienststelle im Sinne des § 9 Absatz 1 Satz 2 LHO. § 16 Absatz 2 Satz 5 gilt entsprechend mit der Maßgabe, dass die Aufgabe der Rektorin oder des Rektors die oder der Vorsitzende des exekutiven Organs nach § 65 a Absatz 3 Satz 4 und die Aufgabe des Hochschulrats das legislative Organ nach § 65 a Absatz 3 Satz 2 wahrnimmt. Die Finanzreferentin oder der Finanzreferent der Studierendenschaft arbeitet mit der oder dem Beauftragten für den Haushalt zusammen. Die Kosten der oder des Beauftragten für den Haushalt trägt die Studierendenschaft. Von Satz 1 kann in begründeten Ausnahmefällen mit Zustimmung des Wissenschaftsministeriums abgewichen werden.

(3) Die Haushalts- und Wirtschaftsführung der Studierendenschaft unterliegt der Prüfung durch den Rechnungshof. Die Studierendenschaft beauftragt zur Rechnungsprüfung darüber hinaus eine fachkundige Person mit der Befähigung für den gehobenen Verwaltungsdienst, die nicht mit der oder dem Beauftragten für den Haushalt gemäß Absatz 2 Satz 1 identisch ist, oder die Verwaltung der Hochschule mit deren Einvernehmen. Die Entlastung erteilt das Rektorat der Hochschule.

(4) Für Verbindlichkeiten haftet die Studierendenschaft mit ihrem Vermögen. Die Hochschule und das Land haften nicht für Verbindlichkeiten der Studierendenschaft.

(5) Studierende, die vorsätzlich oder grob fahrlässig die ihnen obliegenden Pflichten verletzen, insbesondere Gelder der Studierendenschaft für die Erfüllung anderer als der in § 65 Absätze 2 bis 4 genannten Aufgaben verwenden, haben der Studierendenschaft den ihr daraus entstehenden Schaden zu ersetzen. Für die Verjährung von Ansprüchen der Studierendenschaft gelten § 59 LBG und § 48 BeamtStG entsprechend.

(6) Die Studierendenschaft untersteht der Rechtsaufsicht des Rektorats der Hochschule. Für die Rechtsaufsicht gelten § 67 Absatz 1 und § 68 Absätze 1, 3 und 4 entsprechend; die Aufgabe des Wissenschaftsministeriums übernimmt das Rektorat der Hochschule. Die Satzungen und der Haushaltsplan bedürfen der Genehmigung des Rektorats der Hochschule. Die Genehmigung darf nur versagt werden, wenn die Satzung oder der Haushaltsplan rechtswidrig ist. An der DHBW kann das Rektorat die Rechtsaufsicht über die Studierendenvertretung nach § 65 a Absatz 4 Satz 4 generell oder im Einzelfall auf die Rektorin oder den Rektor der Studienakademie übertragen.

(7) Eine wirtschaftliche Betätigung der Studierendenschaft ist nur innerhalb der ihr obliegenden Aufgaben und nur insoweit zulässig, als die Betätigung nach Art und Umfang in einem angemessenen Verhältnis zur Leistungsfähigkeit der Studierendenschaft und zum voraussichtlichen Bedarf steht. Darlehen darf die Studierendenschaft nicht aufnehmen oder vergeben; sie darf ein Girokonto auf Guthabenbasis führen. Die Beteiligung der Studierendenschaft an wirtschaftlichen Unternehmen oder die Gründung wirtschaftlicher Unternehmen bedarf der vorherigen Zustimmung des Rektorats der Hochschule.



%filler
%\newpage
%\thispagestyle{empty}
%\null
%\vfill
%\begin{center}
%	\textit{StuVe, uulm, 2013}
%\end{center}


\end{document}
